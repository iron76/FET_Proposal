\providecommand{\classoptions}{keys}
\documentclass[noworkareas,deliverables,\classoptions]{euproposal}       % for writing
%\documentclass[submit,noworkareas,deliverables]{euproposal}        % for submission
%\documentclass[submit,public,noworkareas,deliverables]{euproposal} % for public version

\usepackage[utf8]{inputenc}

\usepackage{float}  % used to suppress floating of tables in Resources section.
\usetikzlibrary{calc,fit,positioning,shapes,arrows,snakes}
\usepackage[normalem]{ulem}
\usepackage{caption}
\usepackage{eurosym}
\usepackage{footnote}


\addbibresource{kwarc.bib}
%%% institutions
\WAinstitution[id=PAR1,
        countryshort=BE,
        acronym=Partner1]
        {Partner One}

\WAinstitution[id=PAR2,
        countryshort=BE,
        acronym=Partner2]
        {Partner Two}

\WAinstitution[id=PAR3,
        countryshort=BE,
        acronym=Partner3]
        {Partner Three}

\WAinstitution[id=PAR4,
        countryshort=BE,
        acronym=Partner4]
        {Partner Four}

\WAperson[id=PAR1P1,
           personaltitle=Prof. ,
           birthdate=1 Jan. 2000,
           academictitle=Professor,
           affiliation=PAR1,
           department=Department for Research,
           privaddress=None of your business,
           privtel=that neither,
           email=p1@par1.com,
           workaddress={Campus, 1, City},
           worktel=+1 234 56 78,
           workfax=N/A
           ]
           {Person 1}

%%% Local Variables: 
%%% mode: latex
%%% TeX-master: "proposal"
%%% End: 
 % Some sections of the included files depend on this.
\usepackage{comments}
\usepackage{framed}

% This is to reflect the EU template numbering scheme
\renewcommand*\thesection{\arabic{section}}

\begin{document}

\phantom{ }

\begin{proposal}[
  % These PM numbers (person months) are for the coordinator to help planning
  % Participants should not change these, but add PM numbers in the CVS in
  % the site descriptions at CVs/*.tex
  site=PAR1,
  site=PAR2,
  site=PAR3,
  site=PAR4,
  botupPM, % we want to work via bottom up PM distribution,
  coordinator=PAR1P1,
  coordinatorsite=PAR1,
  acronym={My-Prj-Acr},
  acrolong={My-Project-Acronym},
  title=My H2020 research project,
  callname=Call: Information and communication technologies,
  callid=Call identifier: H2020-ICT-2016-2017,
  keywords={science, research, ideas},
  instrument=Topic: Advanced robot capabilities research and take-up,
  challengeid =Challenge identifier: ICT-25-2016-2017,
  months=48,
  compactht]
\newcommand{\TheProject}{\pn}% \pn is defined automatically

\begin{center}
Proposal template (technical annex)
\end{center}
\begin{center}
Research and Innovation actions Innovation actions
\end{center}

Please follow the structure of this template when preparing your proposal. It has been designed to ensure that the important aspects of your planned work are presented in a way that will enable the experts to make an effective assessment against the evaluation criteria. Sections 1, 2 and 3 each correspond to an evaluation criterion for a full proposal.

Please be aware that proposals will be evaluated as they were submitted, rather than on their potential if certain changes were to be made. This means that only proposals that successfully address all the required aspects will have a chance of being funded. There will be no possibility for significant changes to content, budget and consortium composition during grant preparation.

\textbf{First stage proposals:} In two-stage submission schemes, at the first stage you only need to complete the parts indicated by a bracket (i.e. \}). These are in the cover page, and sections 1 and 2.

\textbf{Page limit:} \uline{For full proposals, the cover page, and sections 1, 2 and 3, together should not be longer than 70 pages}. All tables in these sections must be included within this limit. The minimum font size allowed is 11 points. The page size is A4, and all margins (top, bottom, left, right) should be at least 15 mm (not including any footers or headers).

\uline{The page limit for a first stage proposal is 10 pages.}
The page limit will be applied automatically; therefore you must remove this instruction page before submitting.

If you attempt to upload a proposal longer than the specified limit, before the deadline you will receive an automatic warning, and will be advised to shorten and re-upload the proposal. After the deadline, any excess pages will be overprinted with a 'watermark', indicating to evaluators that these pages must be disregarded.

Please do not consider the page limit as a target! It is in your interest to keep your text as concise as possible, since experts rarely view unnecessarily long proposals in a positive light.

\clearpage

\begin{abstract}
This was not in the EU community template ...
\end{abstract}


\ifsubmit\else\setcounter{tocdepth}{4}\fi
\tableofcontents
\clearpage

\TOWRITE{PAR1P1}{Larger table of participants}
\TOWRITE{PAR1P1}{Abstract in the first page?}
\TOWRITE{PAR1P1}{Centering}

\begin{draft}
\section*{Things to do \dots}
\TOWRITE{All}{Request from PAR1P1}
\subsection*{Things PAR1P1 asks as the consortium to do}
\begin{verbatim}
- [ ] Do things.
\end{verbatim}
\end{draft}


% ---------------------------------------------------------------------------
%  Section 1: Excellence
% ---------------------------------------------------------------------------

\section{Excellence}
%!TEX root =  ./proposal.tex

\eucommentary{
  \begin{compactitem}
  \item \textbf{Your proposal must address a work programme topic for this call for proposals.}
  \item \textbf{\emph{This section of your proposal will be assessed only to the extent that it is relevant to that topic.}}
  \end{compactitem}
}

\textbf{}



\subsection{Objectives}\label{sec:objectives}

\eucommentary{
  \begin{compactitem}
  \item Describe the specific objectives for the project\footnote{The term 'project' used in this template equates to an 'action' in certain other Horizon 2020 documentation.}, which should be clear, measurable, realistic and achievable within the duration of the project. Objectives should be consistent with the expected exploitation and impact of the project (see section 2).
  \end{compactitem}
}

\subsection{Relation to the work programme}\label{sec:relation-wp}

\eucommentary{
  \begin{compactitem}
  \item Indicate the work programme topic to which your proposal relates, and explain how your proposal addresses the specific challenge and scope of that topic, as set out in the work programme.
  \end{compactitem}
}

\subsection{Concept and methodology}\label{sec:concept_methodology}

\subsubsection*{(a) Concept} \label{sec:concept}

\eucommentary{
  \begin{compactitem}
  \item Describe and explain the overall concept underpinning the project. Describe the main ideas, models or assumptions involved. Identify any inter-disciplinary considerations and, where relevant, use of stakeholder knowledge;
  \item Describe the positioning of the project e.g. where it is situated in the spectrum from 'idea to application', or from 'lab to market'. Refer to Technology Readiness Levels where relevant. (See General Annex G of the work programme);  
  \item Describe any national or international research and innovation activities which will be linked with the project, especially where the outputs from these will feed into the project;
  \end{compactitem}
}

\subsubsection*{(b) Methodology} \label{sec:methodology}

\eucommentary{
  \begin{compactitem}
  \item Describe and explain the overall methodology, distinguishing, as appropriate, activities indicated in the relevant section of the work programme, e.g. for research, demonstration, piloting, first market replication, etc;
  \item Where relevant, describe how sex and/or gender analysis is taken into account in the project's content.
  \end{compactitem}}

\eucommentary{\emph{Sex and gender refer to biological characteristics and social/cultural factors respectively. For guidance on methods of sex/gender analysis and the issues to be taken into account, please refer to \url{http://ec.europa.eu/research/swafs/gendered-innovations/index_en.cfm?pg=home}}
}

\subsection{Ambition}\label{sec:ambition}
\eucommentary{
  \begin{compactitem}
  \item Describe the advance your proposal would provide beyond the state-of-the-art, and the extent the proposed work is ambitious.
  \item Describe the innovation potential \textbf{(e.g. ground-breaking objectives, novel concepts and approaches, new products, services or business and organisational models)} which the proposal represents. Where relevant, refer to products and services already available on the market. Please refer to the results of any patent search carried out.
  \end{compactitem}
}

%%% Local Variables:
%%% mode: latex
%%% TeX-master: "proposal"
%%% End:


% ---------------------------------------------------------------------------
%  Section 2: Impact
% ---------------------------------------------------------------------------

\section{Impact}
%!TEX root =  ./proposal.tex

\subsection{Expected impacts}

\eucommentary{
  Please be specific, and provide only information that applies to the proposal and its objectives. Wherever possible, use quantified indicators and targets.}


\eucommentary{
  \begin{compactitem}
  \item Describe how your project will contribute to:
  \begin{compactitem}
  \item each of the expected impacts mentioned in the work programme, under the relevant topic;
  \item any substantial impacts not mentioned in the work programme, that would enhance innovation capacity; create new market opportunities, strengthen competitiveness and growth of companies, address issues related to climate change or the environment, or bring other important benefits for society
   \end{compactitem}
  \item Describe any barriers/obstacles, and any framework conditions (such as regulation, standards, public acceptance, workforce considerations, financing of follow-up steps, cooperation of other links in the value chain), that may determine whether and to what extent the expected impacts will be achieved. (This should not include any risk factors concerning implementation, as covered in section 3.2.)
  \end{compactitem}}

\subsection{Measures to maximise impact}

\subsubsection*{a) Dissemination and exploitation of results}

\eucommentary{
  \begin{compactitem}
  \item Provide a draft 'plan for the dissemination and exploitation of the project's results'. Please note that such a draft plan is an admissibility condition, unless the work programme topic explicitly states that such a plan is not required. 
  \item Show how the proposed measures will help to achieve the expected impact of the project.
  \item The plan, should be proportionate to the scale of the project, and should contain measures to be implemented both during and after the end of the project. For innovation actions, in particular, please describe a credible path to deliver these innovations to the market.
  \end{compactitem}
}


\eucommentary{
Your plan for the dissemination and exploitation of the project's results is key to maximising their \textbf{impact}. This plan should describe, in a concrete and comprehensive manner, the \textbf{area} in which you expect to make an impact and \textbf{who} are the potential users of your results. Your plan should also describe \textbf{how} you intend to use the appropriate channels of dissemination and interaction with potential users.}


\eucommentary{
Consider the full range of potential users and uses, including research, commercial, investment, social, environmental, policy-making, setting standards, skills and educational training where relevant.}


\eucommentary{
Your plan should give due consideration to the possible \textbf{follow-up} of your project, once it is finished. Its exploitation could require additional investments, wider testing or scaling up. Its exploitation could also require other pre-conditions like regulation to be adapted, or value chains to adopt the results, or the public at large being receptive to your results.}


\eucommentary{
  \begin{compactitem}
  \item Include a business plan where relevant.  
  \item If you will take part in the pilot on Open Research Data\footnote{Certain actions under Horizon 2020 participate in the 'Pilot on Open Research Data in Horizon 2020'. All other actions can participate on a voluntary basis to this pilot. Further guidance is available in the H2020 Online Manual on the Participant Portal.}, include information on how the participants will manage the research data generated and/or collected during the project, in particular addressing the following issues\footnote{For further guidance on research data management, please refer to the H2020 Online Manual on the Participant Portal.}:  
  \begin{compactitem}
  \item What types of data will the project generate/collect?  
  \item What standards will be used?  
  \item How will this data be exploited and/or shared/made accessible for verification and re-use? If data cannot be made available, explain why.  
  \item How will this data be curated and preserved?  
  \end{compactitem}
  \end{compactitem}}


\eucommentary{
You will need an appropriate consortium agreement to manage (amongst other things) the ownership and access to key knowledge (IPR, data etc.). Where relevant, these will allow you, collectively and individually, to pursue market opportunities arising from the project's results.}


\eucommentary{
The appropriate structure of the consortium to support exploitation is addressed in section 3.3.}


\eucommentary{
  \begin{compactitem}
  \item Outline the strategy for \textbf{knowledge management and protection}. Include measures to provide \textbf{open access} (free on-line access, such as the 'green' or 'gold' model) to peer- reviewed scientific publications which might result from the project\footnote{Open access must be granted to all scientific publications resulting from Horizon 2020 actions. Further guidance on ppen access is available in the H2020 Online Manual on the Participant Portal.}.
  \end{compactitem}}


\eucommentary{
Open access publishing (also called 'gold' open access) means that an article is immediately provided in open access mode by the scientific publisher. The associated costs are usually shifted away from readers, and instead (for example) to the university or research institute to which the researcher is affiliated, or to the funding agency supporting the research.}


\eucommentary{
Self-archiving (also called 'green' open access) means that the published article or the final peer- reviewed manuscript is archived by the researcher - or a representative - in an online repository before, after or alongside its publication. Access to this article is often - but not necessarily - delayed ('embargo period'), as some scientific publishers may wish to recoup their investment by selling subscriptions and charging pay-per-download/view fees during an exclusivity period}

\subsubsection*{b) Communication activities\footnote{Not applicable to SME Instrument, phase 1.}}

\eucommentary{
  \begin{compactitem}
  \item Describe the proposed communication measures for promoting the project and its findings during the period of the grant\footnote{For further guidance on communicating EU research and innovation guidance for project participants, please refer to the H2020 Online Manual on the Participant Portal.}. Measures should be proportionate to the scale of the project, with clear objectives. They should be tailored to the needs of different target audiences, including groups beyond the project's own community. Where relevant, include measures for public/societal engagement on issues related to the project.
  \end{compactitem}
}

%%% Local Variables:
%%% mode: latex
%%% TeX-master: "proposal"
%%% End:


% ---------------------------------------------------------------------------
%  Section 3: Implementation
% ---------------------------------------------------------------------------

\section{Implementation}

\gantttaskchart[draft,xscale=.23,yscale=.33,milestones]

%!TEX root =  ./proposal.tex

\subsection{Project work plan}
\label{sec:wp}

\eucommentary{Please provide the following:
\begin{compactitem}
\item brief presentation of the overall structure of the work plan;
\item timing of the different work packages and their components (Gantt chart or
  similar);
\item detailed work description, i.e.:
  \begin{compactitem}
  \item a description of each work package (table 3.1a);
  \item a list of work packages (table 3.1b);
  \item a list of major deliverables (table 3.1c);
  \end{compactitem}
\item graphical presentation of the components showing how they inter-relate (Pert
  chart or similar).
\end{compactitem}}


\eucommentary{Give full details. Base your account on the logical structure of the project and
the stages in which it is to be carried out. Include details of the resources to
be allocated to each work package. The number of work packages should be
proportionate to the scale and complexity of the project.}


\eucommentary{You should give enough detail in each work package to justify the proposed
resources to be allocated and also quantified information so that progress can
be monitored, including by the Commission.}


\eucommentary{You are advised to include a distinct work package on 'management' (see section 3.2) and to give due visibility in the work plan to 'dissemination and exploitation' and 'communication activities', either with distinct tasks or distinct work packages.}


\eucommentary{You will be required to include an updated (or confirmed) 'plan for the
dissemination and exploitation of results' in both the periodic and final
reports. (This does not apply to topics where a draft plan was not required.)
This should include a record of activities related to dissemination and
exploitation that have been undertaken and those still planned. A report of
completed and planned communication activities will also be required.}


\eucommentary{If your project is taking part in the Pilot on Open Research Data, you must include a 'data management plan' as a distinct deliverable within the first 6 months of the project. A template for such a plan is given in the guidelines on data management in the H2020 Online Manual. This deliverable will evolve during the lifetime of the project in order to present the status of the project's reflections on data management.}


\eucommentary{\\
\textbf{Definitions:}
\begin{compactitem}
\item \uline{'Work package'} means a major sub-division of the proposed project.
\item \uline{'Deliverable'} means a distinct output of the project, meaningful in terms of the project's overall objectives and constituted by a report, a document, a technical diagram, a software etc.
\end{compactitem}
}

\subsection{Management structure, milestones and procedures} \label{sec:management}

\eucommentary{
  \begin{compactitem}
  \item Describe the organisational structure and the decision-making (including a list of milestones (table 3.2a)) .
\item Explain why the organisational structure and decision-making mechanisms are appropriate to the complexity and scale of the project.
\item Describe, where relevant, how effective innovation management will be addressed in the management structure and work plan.
  \end{compactitem}}
  
\eucommentary{  
\emph{Innovation management is a process which requires an understanding of both market and technical problems, with a goal of successfully implementing appropriate creative ideas. A new or improved product, service or process is its typical output. It also allows a consortium to respond to an external or internal opportunity.}}

\eucommentary{
  \begin{compactitem}
  \item Describe any critical risks, relating to project implementation, that the stated project's objectives may not be achieved. Detail any risk mitigation measures. Please provide a table with critical risks identified and mitigating actions (table 3.2b)
\end{compactitem}}


\eucommentary{\\
\textbf{Definitions:}
\begin{compactitem}
\item[] \uline{'Milestones'} means control points in the project that help to chart progress. Milestones may correspond to the completion of a key deliverable, allowing the next phase of the work to begin. They may also be needed at intermediary points so that, if problems have arisen, corrective measures can be taken. A milestone may be a critical decision point in the project where, for example, the consortium must decide which of several technologies to adopt for further development.
\end{compactitem}
}

\subsection{Consortium as a whole}

\eucommentary{The individual members of the consortium are described in a
  separate section 4. There is no need to repeat that information here.\\
%
  \begin{compactitem}
  \item Describe the consortium. How will it match the project?s objectives, and bring together the necessary expertise? How do the members complement one another (and cover the value chain, where appropriate),?
  \item In what way does each of them contribute to the project? Show that each has a valid role, and adequate resources in the project to fulfil that role.
  \item If applicable, describe the industrial/commercial involvement in the project to ensure exploitation of the results and explain why this is consistent with and will help to achieve the specific measures which are proposed for exploitation of the results of the project (see section 2.2).
  \item \textbf{Other countries and international organisations:} If one or more of the participants requesting EU funding is based in a country or is an international organisation that is not automatically eligible for such funding (entities from Member States of the EU, from Associated Countries and from one of the countries in the exhaustive list included in General Annex A of the work programme are automatically eligible for EU funding), explain why the participation of the entity in question is essential to carrying out the project
  \end{compactitem}
}

\subsection{Resources to be committed} \label{sec:resources}

\eucommentary{
  Please make sure the information in this section matches the costs as stated
  in the budget table in section 3 of the administrative proposal forms, and the
  number of person/months, shown in the detailed work package descriptions.}

\eucommentary{
  \begin{compactitem}
  \item a table showing number of person/months required (table 3.4a)
  \item a table showing 'other direct costs' (table 3.4b) for participants where
    those costs exceed 15\% of the personnel costs (according to the budget
    table in section 3 of the administrative proposal forms)
  \end{compactitem}
}


% ---------------------------------------------------------------------------
% Include Work package descriptions
% ---------------------------------------------------------------------------
\newpage
%!TEX root =  ../proposal.tex


\subsubsection*{Tables for section \ref*{sec:wp}}\label{sec:workpackages}
\subsubsection*{Table \ref*{sec:wp}.a: Work package description}\label{sec:workpackages_description}
%% WP titles and order are defined in deliverables.tex
%%% work package style may be broken -- fix this!!

%% Local WP number counter - should possibly be global and hidden?
\begin{workplan}
%!TEX root =  ../proposal.tex

\begin{workpackage}[id=WP1,wphases=0-48,
  short=First WP,% for Figure 5.
  title=First Work Package,
  lead=PAR1,
  PAR1RM=12,
  PAR2RM=6,
  PAR3RM=24]

\begin{wpobjectives}
  There are objectives.
  \begin{compactitem}
  \item Item 1.
  \item Item 2.
  \end{compactitem}
\end{wpobjectives}

\begin{wpdescription}
\eucommentary{Description of work (where appropriate, broken down into tasks), lead partner and role of participants.}
\end{wpdescription}

\begin{tasklist}

  \begin{task}[title=TASK1,id=task1,PM=15,lead=PAR1,wphases=0-30!0.5]

    First task of the project.
    
  \end{task}

  \begin{task}[title=TASK2,id=task2,PM=15,lead=PAR2,wphases=12-42!0.5]

    Second task of the project.
    
  \end{task}

\end{tasklist}

\begin{wpdelivs}
  \begin{wpdeliv}[due=12,id=mydeliv1,dissem=PU,nature=DEM,lead=PAR1]
      {First deliverable, after 1 year.}
  \end{wpdeliv}
  \begin{wpdeliv}[due=24,id=mydeliv2,dissem=PU,nature=DEM,lead=PAR2]
      {Second deliverable, after 2 years.}
\end{wpdeliv}
\end{wpdelivs}

\end{workpackage}

%\input{WorkPackages/Management}
\end{workplan}

\clearpage
\subsubsection*{Table \ref*{sec:wp}.b: List of work packages}\label{sec:workpackages_list}
\wpfig[label=fig:staffeffort,caption=Summary of Staff Efforts]

\clearpage
\subsubsection*{Table \ref*{sec:wp}.c: List of Deliverables\footnote{If your action is taking part in the Pilot on Open Research Data, you must include a data management plan as a distinct deliverable within the first 6 months of the project. This deliverable will evolve during the lifetime of the project in order to present the status of the project's reflections on data management. A template for such a plan is available on the Participant Portal (Guide on Data Management).}}\label{sec:deliverables}
\inputdelivs{9.3cm}

\eucommentary{
\\
\textbf{KEY}\\
Deliverable numbers in order of delivery dates. Please use the numbering convention $\langle$WP number$\rangle$.$\langle$number of deliverable within that WP$\rangle$.
\\
\\
For example, deliverable 4.2 would be the second deliverable from work package 4.
\\
\\
\textbf{Type:}\\
Use one of the following codes:
\begin{itemize}
\item[] R: Document, report (excluding the periodic and final reports) 
\item[] DEM: Demonstrator, pilot, prototype, plan designs
\item[] DEC: Websites, patents filing, press \& media actions, videos, etc. 
\item[] OTHER: Software, technical diagram, etc.
\end{itemize}
\textbf{Dissemination level:}\\
Use one of the following codes:
\begin{itemize}
\item[] PU: Public, fully open, e.g. web
\item[] CO: Confidential, restricted under conditions set out in Model Grant Agreement
\item[] CI: Classified, information as referred to in Commission Decision 2001/844/EC.
\end{itemize}
\textbf{Delivery date:}\\
Measured in months from the project start date (month 1).}

%%% Local Variables:
%%% mode: latex
%%% TeX-master: "../proposal"
%%% End:


% ---------------------------------------------------------------------------
% Milestones
% ---------------------------------------------------------------------------
%!TEX root =  ./proposal.tex

\begin{milestones}
  \milestone[id=start,month=12,
  verif={Completed all corresponding deliverables and reported the progress in the 2nd Project meeting report.}]
  {Starting the project}
  {This is the first milestone.}

  \milestone[id=mymile1,month=24,
  verif={Completed all corresponding deliverables and reported the progress in the 4th Project meeting report.}]
  {Another milestone}
  {By this milestone the project will be in good way.}

  \milestone[id=mymile2,month=36,
  verif={Completed all corresponding deliverables and reported the progress in the 6th Project meeting report.}]
  {Additional milestone}
  {By this milestone the project will be almost over.}

  \milestone[id=finalmile,month=48,
  verif={Completed all corresponding deliverables and reported the progress in the 8th Project meeting report.}]
  {Final milestone}
  {Celebrating the end of the project.}
\end{milestones}



%%% Local Variables:
%%% mode: latex
%%% TeX-master: "proposal"
%%% End:


\newpage
\subsubsection*{Tables for section \ref*{sec:management}}\label{sec:milestones}
\subsubsection*{Table \ref*{sec:management}.a: List of milestones}\label{sec:milestones_list}
\milestonetable
\eucommentary{KEY:
\\
\textbf{Due date}\\ 
Measured in months from the project start date (month 1)\\
\\
\noindent \textbf{Means of verification}\\
Show how you will confirm that the milestone has been attained. Refer to indicators if appropriate. For example: a laboratory prototype that is 'up and running'; software released and validated by a user group; field survey complete and data quality validated.}

% ---------------------------------------------------------------------------
% Risks
% ---------------------------------------------------------------------------

\subsubsection*{Table \ref*{sec:management}.b: Critical risks for implementation}
%!TEX root =  ./proposal.tex

\begin{table}[ht]
\begin{tabular}{|p{5.5cm}|p{3cm}|p{6.5cm}|}\hline
\textbf{Description of risk (indicate level of likelihood: Low/Medium/High)} & \textbf{Work package(s) involved} & \textbf{Proposed risk-mitigation measures} \\ \hline\hline
 & &  \\
 & &  \\
 & &  \\ \hline
\end{tabular}
\caption*{Inserte caption hree}
\end{table}

\eucommentary{\\
\textbf{Definition critical risk}:\\
A critical risk is a plausible event or issue that could have a high adverse impact on the ability of the project to achieve its objectives.
\\
\\
\textbf{Level of likelihood} to occur: \textbf{Low/medium/high}\\
The likelihood is the estimated probability that the risk will materialise even after taking account of the mitigating measures put in place.}

% ---------------------------------------------------------------------------
% Staff effort
% ---------------------------------------------------------------------------
\newpage
\subsubsection*{Table for section \ref*{sec:resources}}
\subsubsection*{Table \ref*{sec:resources}.a: Summary of staff effort}

\eucommentary{Please indicate the number of person/months over the whole duration of the planned work, for each work package, for each participant. Identify the work-package leader for each WP by showing the relevant person-month figure in bold.}

\wpfig[label=fig:staffeffort2,caption=Summary of Staff Efforts]

\subsubsection*{Table \ref*{sec:resources}.b: 'Other direct cost' items (travel, equipment, other goods and services, large research infrastructure)}

\eucommentary{Please complete the table below for each participant if the sum of the costs for 'travel', 'equipment', and 'goods and services' exceeds 15\% of the personnel costs for that participant (according to the budget table in section 3 of the proposal administrative forms).}

\begin{table}[ht]
\begin{tabular}{|p{3.5cm}|p{2cm}|p{9.5cm}|}\hline
\textbf{Participant Number/Short Name} & \textbf{Cost (\euro{})} & \textbf{Justification} \\ \hline\hline
 \textbf{Travel} & &  \\ \hline
 \textbf{Equipment} & &  \\ \hline
 \textbf{Other goods and services} & &  \\ \hline
  \textbf{Total} & &  \\ \hline
\end{tabular}
\caption*{Insert the caption here.}
\end{table}

\eucommentary{Please complete the table below for all participants that would like to declare costs of large research infrastructure under Article 6.2 of the General Model Agreement\footnote{Large research infrastructure means research infrastructure of a total value of at least EUR 20 million, for a beneficiary. More information and further guidance on the direct costing for the large research infrastructure is available in the H2020 Online Manual on the Participant Portal.}, irrespective of the percentage of personnel costs. Please indicate (in the justification) if the beneficiary?s methodology for declaring the costs for large research infrastructure has already been positively assessed by the Commission.}

\begin{table}[ht]
\begin{tabular}{|p{3.5cm}|p{2cm}|p{9.5cm}|}\hline
\textbf{Participant Number/Short Name} & \textbf{Cost (\euro{})} & \textbf{Justification} \\ \hline\hline
 \textbf{Large research infrastructure} & &  \\ \hline
\end{tabular}
\caption*{Insert the caption here.}
\end{table}

% ---------------------------------------------------------------------------
% Consortium
% ---------------------------------------------------------------------------

\newpage
%!TEX root =  ./proposal.tex

\section{Members of the consortium}

\eucommentary{This section is not covered by the page limit.}
\noindent
\eucommentary{The information provided here will be used to judge the operational capacity.}

\eucommentary{\begin{compactitem}
\item
Describe the consortium. How will it match the project's objectives?
How do the members complement one another (and cover the value chain,
where appropriate)? In what way does each of them contribute to the
project? How will they be able to work effectively together?
\item
If applicable, describe the industrial/commercial involvement in the
project to ensure exploitation of the results and explain why this is
consistent with and will help to achieve the specific measures which
are proposed for exploitation of the results of the project (see section 2.3).
\item
Other countries: If one or more of the participants requesting EU funding
is based in a country that is not automatically eligible for such funding
(entities from Member States of the EU, from Associated Countries and
from one of the countries in the exhaustive list included in General
Annex A of the work programme are automatically eligible for EU funding),
 explain why the participation of the entity in question is essential to carrying out the project
\end{compactitem}}

\subsection{Participants (applicants)}

\eucommentary{Please provide, for each participant, the following (if available):
\begin{compactitem}
\item a description of the legal entity and its main tasks, with an explanation of how its profile matches the tasks in the proposal;
\item a curriculum vitae or description of the profile of the persons, including their gender, who will be primarily responsible for carrying out the proposed research and/or innovation activities;
\item a list of up to 5 relevant publications, and/or products, services (including widely- used datasets or software), or other achievements relevant to the call content;
\item a list of up to 5 relevant previous projects or activities, connected to the subject of this proposal;
\item a description of any significant infrastructure and/or any major items of technical equipment, relevant to the proposed work;
\item any other supporting documents specified in the work programme for this call.
\end{compactitem}}

\input{Participants/PAR1.tex}
\clearpage
\input{Participants/PAR2.tex}
\clearpage
\input{Participants/PAR3.tex}
\clearpage


%%% Local Variables:
%%% mode: latex
%%% TeX-master: "proposal"
%%% End:


\draftpage
%!TEX root =  ./proposal.tex

\paragraph{Resources PAR1}

PAR1 will consist of PAR1P1 and PAR1P2.

\paragraph{Resources PAR2}

PAR2 will consist of PAR2P1.

\paragraph{Resources PAR3}

PAR1 will consist of PAR3P1 and PAR3P2.

\subsection{Third parties involved in the project (including use of third party resources)}

\eucommentary{Please complete, for each participant, the following table (or simply state "No third parties involved", if applicable):}

\begin{savenotes}
\begin{table}[ht]
\begin{tabular}{|p{13cm}|p{2cm}|}\hline
Does the participant plan to subcontract certain tasks (please note that core tasks of the project should not be sub-contracted) & Y/N \\ \hline
\multicolumn{2}{|p{15cm}|}{\textit{If yes, please describe and justify the tasks to be subcontracted}} \\ \hline
%
Does the participant envisage that part of its work is performed by linked third parties\footnote{A third party that is an affiliated entity or has a legal link to a participant implying a collaboration not limited to the action. (Article 14 of the Model Grant Agreement).} & Y/N \\ \hline
\multicolumn{2}{|p{15cm}|}{\textit{If yes, please describe the third party, the link of the participant to the third party, and describe and justify the foreseen tasks to be performed by the third party}} \\ \hline
%
Does the participant envisage the use of contributions in kind provided by third parties (Articles 11 and 12 of the General Model Grant Agreement) & Y/N \\ \hline
\multicolumn{2}{|p{15cm}|}{\textit{If yes, please describe the third party and their contributions}} \\ \hline
\end{tabular}
\caption*{Insert the caption here.}
\end{table}
\end{savenotes}

%%%%%%%%%%%%%%%%%%%%%%%%%%%%%%%%%%%%%%%%%%%%%%%%%%%%%%%%%%%%%%%%
%
% Guidelines for completion of partner specific resource summary:
%
%
% Please explain how many person months for each person are
% requested. Say who is the local lead. Say anything that helps to
% understand why people are recruited as you plan, in particular if
% this deviates from having one research for 48 months.  We can also
% use this bit of the proposal (and the table, see below) to address
% any other unusual arrangements.
%
%
% The table should contain all non-staff costs (the EU requests that
% this table must be present if the non-staff costs exceed
% 15% of the total cost, but it is good practice and will show
% openness and transparency that we show the data for all partners).
%
% Link back from the table to the work packages and tasks for which
% the expenses are required. Add information that makes it easier to
% understand why the expenses are justified.
%
%     To refer to a task in a work package, use "\taskref{WP-ID}{TASK-ID}" where
%     WP-ID is the ID of the work package:
%        WP#: WP-ID - full title
%        ----------------------
%        WP1: 'management' - Management
%        WP2: 'community' - Community Building and Engagement
%        WP3: 'component-architecture' - Component Architecture
%        WP4: 'UI' - User interfaces
%        WP5: 'hpc' - High Performance Computing
%        WP6: 'dksbases' - Data/Knowledge/Software-Bases
%        WP7: 'social-aspects' - Social Aspects
%        WP8: 'dissem' - Dissemination
%
%
%     and "TASK-ID" is the ID of the task. You can set this using
%
%       \begin{task}[id=TASK-ID,title=Math Search Engine,lead=JU,PM=10,lead=JU]
%
%     To refer to deliverables, use "\delivref{WP-ID}{DELIV-ID}" where DELIV-ID is
%     the ID of the deliverable that can be set like this:
%
%       \begin{wpdeliv}[due=36,id=DELIV-ID,dissem=PU,nature=DEM]
%           {Exploratory support for semantic-aware interactive widgets providing views on objects
%           represented and or in databases}
%       \end{wpdeliv}
%
%
% The table is pre-populated with entries most sites are likely
% to need. If a line does not apply to you, just delete it. If you need
% an extra line, then add it. Use common sense: the number of rows should not
% be very big, but at the same time it is useful to give some breakdown/explanation
% of costs.
%
%
% Eventually, try to create you entry similar in style to the others.
% (The Southampton entry is fully populated, so use this as guidance
% if in doubt.)
%
%
%%%%%%%%%%%%%%%%%%%%%%%%%%%%%%%%%%%%%%%%%%%%%%%%%%%%%%%%%%%%%%%%

%%%%%%%%%%%%%%%%%%%%%%%%%%%%%%%%%%%%%%%%%%%%%%%%%%%%%%%%%%%%%%%%%%%%%%%%%%%%%%


%%% Local Variables:
%%% mode: latex
%%% TeX-master: "proposal"
%%% End:







%No third parties involved.

% ---------------------------------------------------------------------------
%  Section 5: Ethics and Security
% ---------------------------------------------------------------------------

\newpage

\section{Ethics and Security}

\eucommentary{This section is not covered by the page limit.}

\subsection{Ethics}

\eucommentary{
If you have entered any ethics issues in the ethical issue table in the administrative proposal forms, you must:\\
$\bullet$ submit an ethics self-assessment, which: \\
-- describes how the proposal meets the national legal and ethical requirements of the
country or countries where the tasks raising ethical issues are to be carried out;\\
-- explains in detail how you intend to address the issues in the ethical issues table, in
particular as regards:
research objectives (e.g. study of vulnerable populations, dual use, etc.),
research methodology (e.g. clinical trials, involvement of children and related
consent procedures, protection of any data collected, etc.),
the potential impact of the research (e.g. dual use issues, environmental damage,
stigmatisation of particular social groups, political or financial retaliation,
benefit-sharing, malevolent use , etc.)\\
$\bullet$ provide the documents that you need under national law (if you already have them), e.g.:\\
-- an ethics committee opinion;\\
-- the document notifying activities raising ethical issues or authorizing such activities}

\eucommentary{If these documents are not in English, you must also submit an English summary of them (containing, if available, the conclusions of the committee or authority concerned).}

\eucommentary{If you plan to request these documents specifically for the project
you are proposing, your request must contain an explicit reference to the project title.}

\subsection[Security]{Security\protect\footnote{Article 37.1 of the Model Grant Agreement: Before disclosing results of activities raising security issues to a third party (including affiliated entities), a beneficiary must inform the coordinator ? which must request written approval from the Commission/Agency. Article 37.2: Activities related to ?classified deliverables? must comply with the ?security requirements? until they are declassified. Action tasks related to classified deliverables may not be subcontracted without prior explicit written approval from the Commission/Agency. The beneficiaries must inform the coordinator ? which must immediately inform the Commission/Agency ? of any changes in the security context and ? if necessary ? request for Annex 1 to be amended (see Article 55)}} 

Please indicate if your proposal will involve:

\begin{compactitem}
\item activities or results raising security issues: NO
\item 'EU-classified information' as background or results: NO
\end{compactitem}
\end{proposal}
\phantom{Last page is left blank for a bug on page numbers :)}
\end{document}

%%% Local Variables:
%%% mode: latex
%%% TeX-master: t
%%% End:


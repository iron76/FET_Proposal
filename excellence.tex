%!TEX root =  ./proposal.tex

\eucommentary{
  \begin{compactitem}
  \item \textbf{Your proposal must address a work programme topic for this call for proposals.}
  \item \textbf{\emph{This section of your proposal will be assessed only to the extent that it is relevant to that topic.}}
  \end{compactitem}
}

\textbf{}



\subsection{Objectives}\label{sec:objectives}

\eucommentary{
  \begin{compactitem}
  \item Describe the specific objectives for the project\footnote{The term 'project' used in this template equates to an 'action' in certain other Horizon 2020 documentation.}, which should be clear, measurable, realistic and achievable within the duration of the project. Objectives should be consistent with the expected exploitation and impact of the project (see section 2).
  \end{compactitem}
}

\subsection{Relation to the work programme}\label{sec:relation-wp}

\eucommentary{
  \begin{compactitem}
  \item Indicate the work programme topic to which your proposal relates, and explain how your proposal addresses the specific challenge and scope of that topic, as set out in the work programme.
  \end{compactitem}
}

\subsection{Concept and methodology}\label{sec:concept_methodology}

\subsubsection*{(a) Concept} \label{sec:concept}

\eucommentary{
  \begin{compactitem}
  \item Describe and explain the overall concept underpinning the project. Describe the main ideas, models or assumptions involved. Identify any inter-disciplinary considerations and, where relevant, use of stakeholder knowledge;
  \item Describe the positioning of the project e.g. where it is situated in the spectrum from 'idea to application', or from 'lab to market'. Refer to Technology Readiness Levels where relevant. (See General Annex G of the work programme);  
  \item Describe any national or international research and innovation activities which will be linked with the project, especially where the outputs from these will feed into the project;
  \end{compactitem}
}

\subsubsection*{(b) Methodology} \label{sec:methodology}

\eucommentary{
  \begin{compactitem}
  \item Describe and explain the overall methodology, distinguishing, as appropriate, activities indicated in the relevant section of the work programme, e.g. for research, demonstration, piloting, first market replication, etc;
  \item Where relevant, describe how sex and/or gender analysis is taken into account in the project's content.
  \end{compactitem}}

\eucommentary{\emph{Sex and gender refer to biological characteristics and social/cultural factors respectively. For guidance on methods of sex/gender analysis and the issues to be taken into account, please refer to \url{http://ec.europa.eu/research/swafs/gendered-innovations/index_en.cfm?pg=home}}
}

\subsection{Ambition}\label{sec:ambition}
\eucommentary{
  \begin{compactitem}
  \item Describe the advance your proposal would provide beyond the state-of-the-art, and the extent the proposed work is ambitious.
  \item Describe the innovation potential \textbf{(e.g. ground-breaking objectives, novel concepts and approaches, new products, services or business and organisational models)} which the proposal represents. Where relevant, refer to products and services already available on the market. Please refer to the results of any patent search carried out.
  \end{compactitem}
}

%%% Local Variables:
%%% mode: latex
%%% TeX-master: "proposal"
%%% End:

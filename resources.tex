%!TEX root =  ./proposal.tex

\paragraph{Resources PAR1}

PAR1 will consist of PAR1P1 and PAR1P2.

\paragraph{Resources PAR2}

PAR2 will consist of PAR2P1.

\paragraph{Resources PAR3}

PAR1 will consist of PAR3P1 and PAR3P2.

\subsection{Third parties involved in the project (including use of third party resources)}

\eucommentary{Please complete, for each participant, the following table (or simply state "No third parties involved", if applicable):}

\begin{savenotes}
\begin{table}[ht]
\begin{tabular}{|p{13cm}|p{2cm}|}\hline
Does the participant plan to subcontract certain tasks (please note that core tasks of the project should not be sub-contracted) & Y/N \\ \hline
\multicolumn{2}{|p{15cm}|}{\textit{If yes, please describe and justify the tasks to be subcontracted}} \\ \hline
%
Does the participant envisage that part of its work is performed by linked third parties\footnote{A third party that is an affiliated entity or has a legal link to a participant implying a collaboration not limited to the action. (Article 14 of the Model Grant Agreement).} & Y/N \\ \hline
\multicolumn{2}{|p{15cm}|}{\textit{If yes, please describe the third party, the link of the participant to the third party, and describe and justify the foreseen tasks to be performed by the third party}} \\ \hline
%
Does the participant envisage the use of contributions in kind provided by third parties (Articles 11 and 12 of the General Model Grant Agreement) & Y/N \\ \hline
\multicolumn{2}{|p{15cm}|}{\textit{If yes, please describe the third party and their contributions}} \\ \hline
\end{tabular}
\caption*{Insert the caption here.}
\end{table}
\end{savenotes}

%%%%%%%%%%%%%%%%%%%%%%%%%%%%%%%%%%%%%%%%%%%%%%%%%%%%%%%%%%%%%%%%
%
% Guidelines for completion of partner specific resource summary:
%
%
% Please explain how many person months for each person are
% requested. Say who is the local lead. Say anything that helps to
% understand why people are recruited as you plan, in particular if
% this deviates from having one research for 48 months.  We can also
% use this bit of the proposal (and the table, see below) to address
% any other unusual arrangements.
%
%
% The table should contain all non-staff costs (the EU requests that
% this table must be present if the non-staff costs exceed
% 15% of the total cost, but it is good practice and will show
% openness and transparency that we show the data for all partners).
%
% Link back from the table to the work packages and tasks for which
% the expenses are required. Add information that makes it easier to
% understand why the expenses are justified.
%
%     To refer to a task in a work package, use "\taskref{WP-ID}{TASK-ID}" where
%     WP-ID is the ID of the work package:
%        WP#: WP-ID - full title
%        ----------------------
%        WP1: 'management' - Management
%        WP2: 'community' - Community Building and Engagement
%        WP3: 'component-architecture' - Component Architecture
%        WP4: 'UI' - User interfaces
%        WP5: 'hpc' - High Performance Computing
%        WP6: 'dksbases' - Data/Knowledge/Software-Bases
%        WP7: 'social-aspects' - Social Aspects
%        WP8: 'dissem' - Dissemination
%
%
%     and "TASK-ID" is the ID of the task. You can set this using
%
%       \begin{task}[id=TASK-ID,title=Math Search Engine,lead=JU,PM=10,lead=JU]
%
%     To refer to deliverables, use "\delivref{WP-ID}{DELIV-ID}" where DELIV-ID is
%     the ID of the deliverable that can be set like this:
%
%       \begin{wpdeliv}[due=36,id=DELIV-ID,dissem=PU,nature=DEM]
%           {Exploratory support for semantic-aware interactive widgets providing views on objects
%           represented and or in databases}
%       \end{wpdeliv}
%
%
% The table is pre-populated with entries most sites are likely
% to need. If a line does not apply to you, just delete it. If you need
% an extra line, then add it. Use common sense: the number of rows should not
% be very big, but at the same time it is useful to give some breakdown/explanation
% of costs.
%
%
% Eventually, try to create you entry similar in style to the others.
% (The Southampton entry is fully populated, so use this as guidance
% if in doubt.)
%
%
%%%%%%%%%%%%%%%%%%%%%%%%%%%%%%%%%%%%%%%%%%%%%%%%%%%%%%%%%%%%%%%%

%%%%%%%%%%%%%%%%%%%%%%%%%%%%%%%%%%%%%%%%%%%%%%%%%%%%%%%%%%%%%%%%%%%%%%%%%%%%%%


%%% Local Variables:
%%% mode: latex
%%% TeX-master: "proposal"
%%% End:

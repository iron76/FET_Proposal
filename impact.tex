%!TEX root =  ./proposal.tex

\subsection{Expected impacts}

\eucommentary{
  Please be specific, and provide only information that applies to the proposal and its objectives. Wherever possible, use quantified indicators and targets.}


\eucommentary{
  \begin{compactitem}
  \item Describe how your project will contribute to:
  \begin{compactitem}
  \item each of the expected impacts mentioned in the work programme, under the relevant topic;
  \item any substantial impacts not mentioned in the work programme, that would enhance innovation capacity; create new market opportunities, strengthen competitiveness and growth of companies, address issues related to climate change or the environment, or bring other important benefits for society
   \end{compactitem}
  \item Describe any barriers/obstacles, and any framework conditions (such as regulation, standards, public acceptance, workforce considerations, financing of follow-up steps, cooperation of other links in the value chain), that may determine whether and to what extent the expected impacts will be achieved. (This should not include any risk factors concerning implementation, as covered in section 3.2.)
  \end{compactitem}}

\subsection{Measures to maximise impact}

\subsubsection*{a) Dissemination and exploitation of results}

\eucommentary{
  \begin{compactitem}
  \item Provide a draft 'plan for the dissemination and exploitation of the project's results'. Please note that such a draft plan is an admissibility condition, unless the work programme topic explicitly states that such a plan is not required. 
  \item Show how the proposed measures will help to achieve the expected impact of the project.
  \item The plan, should be proportionate to the scale of the project, and should contain measures to be implemented both during and after the end of the project. For innovation actions, in particular, please describe a credible path to deliver these innovations to the market.
  \end{compactitem}
}


\eucommentary{
Your plan for the dissemination and exploitation of the project's results is key to maximising their \textbf{impact}. This plan should describe, in a concrete and comprehensive manner, the \textbf{area} in which you expect to make an impact and \textbf{who} are the potential users of your results. Your plan should also describe \textbf{how} you intend to use the appropriate channels of dissemination and interaction with potential users.}


\eucommentary{
Consider the full range of potential users and uses, including research, commercial, investment, social, environmental, policy-making, setting standards, skills and educational training where relevant.}


\eucommentary{
Your plan should give due consideration to the possible \textbf{follow-up} of your project, once it is finished. Its exploitation could require additional investments, wider testing or scaling up. Its exploitation could also require other pre-conditions like regulation to be adapted, or value chains to adopt the results, or the public at large being receptive to your results.}


\eucommentary{
  \begin{compactitem}
  \item Include a business plan where relevant.  
  \item If you will take part in the pilot on Open Research Data\footnote{Certain actions under Horizon 2020 participate in the 'Pilot on Open Research Data in Horizon 2020'. All other actions can participate on a voluntary basis to this pilot. Further guidance is available in the H2020 Online Manual on the Participant Portal.}, include information on how the participants will manage the research data generated and/or collected during the project, in particular addressing the following issues\footnote{For further guidance on research data management, please refer to the H2020 Online Manual on the Participant Portal.}:  
  \begin{compactitem}
  \item What types of data will the project generate/collect?  
  \item What standards will be used?  
  \item How will this data be exploited and/or shared/made accessible for verification and re-use? If data cannot be made available, explain why.  
  \item How will this data be curated and preserved?  
  \end{compactitem}
  \end{compactitem}}


\eucommentary{
You will need an appropriate consortium agreement to manage (amongst other things) the ownership and access to key knowledge (IPR, data etc.). Where relevant, these will allow you, collectively and individually, to pursue market opportunities arising from the project's results.}


\eucommentary{
The appropriate structure of the consortium to support exploitation is addressed in section 3.3.}


\eucommentary{
  \begin{compactitem}
  \item Outline the strategy for \textbf{knowledge management and protection}. Include measures to provide \textbf{open access} (free on-line access, such as the 'green' or 'gold' model) to peer- reviewed scientific publications which might result from the project\footnote{Open access must be granted to all scientific publications resulting from Horizon 2020 actions. Further guidance on ppen access is available in the H2020 Online Manual on the Participant Portal.}.
  \end{compactitem}}


\eucommentary{
Open access publishing (also called 'gold' open access) means that an article is immediately provided in open access mode by the scientific publisher. The associated costs are usually shifted away from readers, and instead (for example) to the university or research institute to which the researcher is affiliated, or to the funding agency supporting the research.}


\eucommentary{
Self-archiving (also called 'green' open access) means that the published article or the final peer- reviewed manuscript is archived by the researcher - or a representative - in an online repository before, after or alongside its publication. Access to this article is often - but not necessarily - delayed ('embargo period'), as some scientific publishers may wish to recoup their investment by selling subscriptions and charging pay-per-download/view fees during an exclusivity period}

\subsubsection*{b) Communication activities\footnote{Not applicable to SME Instrument, phase 1.}}

\eucommentary{
  \begin{compactitem}
  \item Describe the proposed communication measures for promoting the project and its findings during the period of the grant\footnote{For further guidance on communicating EU research and innovation guidance for project participants, please refer to the H2020 Online Manual on the Participant Portal.}. Measures should be proportionate to the scale of the project, with clear objectives. They should be tailored to the needs of different target audiences, including groups beyond the project's own community. Where relevant, include measures for public/societal engagement on issues related to the project.
  \end{compactitem}
}

%%% Local Variables:
%%% mode: latex
%%% TeX-master: "proposal"
%%% End:
